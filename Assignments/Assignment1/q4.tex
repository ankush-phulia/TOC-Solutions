\documentclass{article}
\usepackage[utf8]{inputenc}
\usepackage{graphicx}
\usepackage{amsmath}

\title{Question 4}
\author{Raj Kamal (2018CSZ8013) }
\date{12.01.2019}

\begin{document}

\maketitle
We have to give two bijections from set of integers to set of rationals. \newline \newline
\textbf{Bijection 1} \\ \\
For $i~ \epsilon ~\mathbf{N}$, define \\ \\ $T(i)=\left\{
               \begin{array}{ll}
                 \frac{-i}{2}, & \hbox{if i is even;} \\
                 \frac{i+1}{2}, & \hbox{if i is odd.}
               \end{array}
             \right.$
\\ \\ Then $T$ is a bijection from $\mathbf{N}$ to $\mathbf{Z}\setminus\{0\}$
\\ \\If $n = \prod_{k}p_k^{n_k}$
\\ \\define $f: \mathbf{N}\rightarrow \mathbf{Q^+}$ by\\ \\ $f(n)=\prod_{k}p_k^{T(n_k)}$
\\ \\ \textbf{claim:} f is bijection.
\\ \\ \textbf{f is one-one} \\ \\ If $f(m)=f(n)$, we have two rational numbers which are equal and thus have the same fractional representation $\frac{p}{q}$ with $p,~q ~\epsilon~ \mathbf{N}$. Writing the fraction in prime factorization, we get a product like $\prod_{k}p_k^{y_k}$, where $y_k$ are positive or negative integers. We map the $y_k$ to the inverse of the T transformation to get the $n_k$ and thus the $n$. Hence function $f$ is one-one.
\\ \\ \textbf{f is onto} \\ \\ Let $\frac{p}{q}=\prod_{k}p_k^{y_k}$, where $y_k$ are positive or negative integers, be any rational number. \\ \\ Choose $n_k = T^{-1}(y_k)$ \\ \\ If $n=\prod_{k}p_k^{n_k}$ \\ \\ Then we have \\ \\ $f(n)=\prod_{k}p_k^{T(n_k)}=\prod_{k}p_k^{y_k}=\frac{p}{q}$ \\ \\ Therefore there exists a natural number $n$ for each given rational number $\frac{p}{q}$ such that $f(n)=\frac{p}{q}$. \textbf{Hence $f: \mathbf{N}\rightarrow \mathbf{Q^+}$ given by\\ \\ $f(n)=\prod_{k}p_k^{T(n_k)}$ is a bijection.}
\\ \\ Define $g: \mathbf{Z}\rightarrow \mathbf{Q}$ such that \\ \\ $g(n)=\left\{
                                                                          \begin{array}{ll}
                                                                            f(n), & \hbox{if $n>0$;} \\
                                                                            -f(-n), & \hbox{if $n<0$;} \\
                                                                            0, & \hbox{if $n=0$.}
                                                                          \end{array}
                                                                        \right.$
\\ \\ Since $f$ is bijection so $g$ is also a bijection. \\ \\
\textbf{Bijection 2} \\ \\
Note that the set $\mathbf{Q^+}$ of positive rational numbers can be written in array form with repeated values as follows
\begin{center}
\includegraphics[width=3 in]{im1.png}
\end{center}
We map natural numbers diagonally on this array as follows
\begin{center}
\includegraphics[width=3 in]{im2.png}
\end{center}
Numbers in blue are natural numbers. Note that this array contains each natural number exactly once and each positive rational number is image of some natural number so the array given above represents a bijection from natural numbers to positive rational numbers. Let this bijection of natural numbers to set of positive rational numbers be denoted by $f$.
\\ \\ Define $g: \mathbf{Z}\rightarrow \mathbf{Q}$ such that \\ \\ $g(n)=\left\{
                                                                          \begin{array}{ll}
                                                                            f(n), & \hbox{if $n>0$;} \\
                                                                            -f(-n), & \hbox{if $n<0$;} \\
                                                                            0, & \hbox{if $n=0$.}
                                                                          \end{array}
                                                                        \right.$
\\ \\ Since $f$ is bijection so $g$ is also a bijection.
\end{document}
