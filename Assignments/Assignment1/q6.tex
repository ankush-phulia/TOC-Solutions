\documentclass{article}
\usepackage[utf8]{inputenc}
\usepackage{pdfpages}

\usepackage{amsfonts}
\usepackage{amsmath}
\usepackage{graphicx}
\date{January 2019}

\begin{document}

\maketitle
\textbf{Question 6}\\ \\
\textbf{i)}
Since a bijection function is found between the Rationals and Integers in Question 4, thus by the second clause \begin{equation}
    \mathbb{Z} =_{\#} \mathbb{R}
\end{equation}
\\
\textbf{ii)}
Let \begin{math}
f(x): [0,1] \to (10, 100)
\end{math}  be defined as:
\begin{equation}
f(x) = 
     \begin{cases}
       \text{x+11} &\quad\text{if x} \in (0, 1)\\
       \text{12} &\quad\text{if x = 0} \\
       \text{13} &\quad\text{if x = 1} \\
     \end{cases}
\end{equation}
For a one-one function, if \begin{math}
x \neq y
\end{math} then \begin{math}
f(x) \neq f(y)
\end{math}

Case 1:
\begin{math}
x=0, y=1 \\ 
f(x)=12, f(y)=13 \\
f(x) \neq f(y) \\
\end{math}

Case 2: 
\begin{math}
x=0, y \in (0, 1) \\ 
f(x)=12\\
f(y) \in (0+11, 1+11) \implies f(y) \in (11, 12) \\
f(x) \cap f(y) = \phi \\
\end{math}
    

Case 3: 
\begin{math}
x=1, y \in (0, 1) \\ 
f(x)=13\\
f(y) \in (0+11, 1+11) \implies f(y) \in (11, 12) \\
f(x) \cap f(y) = \phi \\
\end{math}
    
Therefore f(x) is a one one function \\

Let \begin{math}
g(x): (10, 100) \to [0, 1]
\end{math}  be defined as:
\begin{equation}
g(x) = 
     \begin{cases}
       \text{x-20} &\quad\text{if x} \in [20, 30]\\
       \text{0} &\quad\text{otherwise} \\
     \end{cases}
\end{equation}

Let 
\begin{math}
y=x-20\end{math} for \begin{math}x \in[20, 30] \\
\implies x=y+20 \end{math}
for every \begin{math}
y \in [0, 1] \\
\implies x \in [0+20, 0+30]  \\
\implies x \in [20, 30]
\end{math}

Thus \begin{math}
g(x)
\end{math} is onto \\

Since \begin{math}
f(x): [0,1] \to (10, 100)
\end{math} is one-one and \begin{math}
g(x): (10, 100) \to [0, 1]
\end{math} is onto, \\therefore \begin{equation}
    [0, 1] <_{\#} (10, 100)
\end{equation} \\

Let \begin{math}
f(x): (10, 100) \to [0, 1]
\end{math}  be defined as:
\begin{equation}
f(x) =  \frac{x-10}{90}
\end{equation}

Let \begin{math} f(x_{1})=f(x_{2}) \end{math} \\
\begin{math}\implies \frac{x_{1}-10}{90}=\frac{x_{2}-10}{90} \end{math} \\
\begin{math}\implies x_{1}=x_{2}\end{math} \\
Therefore \begin{math}
f(x)
\end{math} is one one \\

Let \begin{math}
g(x): [0,1] \to (10, 100)
\end{math}  be defined as:
\begin{equation}
g(x) = 
     \begin{cases}
       \text{\begin{math}
       90*x+10
       \end{math}} &\quad\text{if x} \in (0,1)\\
       \text{20} &\quad\text{if x } \in \{0, 1\} \\
     \end{cases}
\end{equation}
    
Let \begin{math}
y=90*x+10
\end{math} for \begin{math}
x \in (0,1) \\
\implies x=\frac{y-10}{90} \\
\end{math} 
for every \begin{math}
 y \in (10, 100), x \in (\frac{10-10}{90}, \frac{100-10}{90}) \\
\implies x \in (0, 1) \\
\end{math} 
Thus, \begin{math}
g(x)
\end{math} is onto \\

Since \begin{math}
f(x): (10, 100) \to [0, 1]
\end{math} is one-one and \begin{math}
g(x): [0,1] \to (10, 100)
\end{math} is onto, \\ therefore \begin{equation}
(10, 100) <_{\#} [0, 1]
\end{equation} \\

Thus by equation 4 and 7 and Bernstein-Schroeder theorem, \begin{equation}
[0, 1] =_{\#} (10, 100)    
\end{equation}



\end{document}
