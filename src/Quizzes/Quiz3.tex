\documentclass{article}
\usepackage[utf8]{inputenc}
\usepackage[english]{babel}
\usepackage{amssymb}
\usepackage{amsmath}
\usepackage{enumitem}

\usepackage{geometry}
\geometry{left=2.5cm,right=2.5cm,top=2cm,bottom=2cm}

\setlength{\parskip}{0.5em}
\setlength{\parindent}{0em}
\renewcommand{\baselinestretch}{1.0}

\title{Quiz 3}
\author{COL 352\\
    Introduction to Automata \& 
    Theory of Computation}
\date{}

\begin{document}
    \maketitle
    
    \section*{Problem 1} Let $L_H = \{~<M,w>|\ M\ halts\ on\ input\ w \}$. Is $L_H$ recursive, or r.e. but not recursive or not r.e. ? Justify

    \textbf{Solution} : $L_H$ is r.e. but not recursive
    
    Proof :
    
    I) Recursively enumerable : we can construct a Turing Machine $M'$ which semi-decides $L_H$ -
    \begin{enumerate}[topsep=0pt]
    \item  For any input M and w, M' simulates M on w.
    \item  If M halts on input w, go to final state in M'.
    \end{enumerate}
    Hence $L_H$ is recursively enumerable.
    
    II) Not recursive : we know that Universal TM language is an undecidable/non-recursive language.
    
    \quad We can show that $L_H$ is also non-recursive by reducing the Universal TM language ($L_U$) to $L_H$, i.e.
    
    \quad $L_U\ {\le }_f\ L_H$, where $f$ is a Turing computable function and is defined as $f(<M,~w>)=\ <M',~w'>$ 
    
    \quad such that M' halts on w iff M accepts w, i.e.
    \begin{enumerate}[leftmargin=3em, topsep=0pt]
    \item  First simulate M on w.
    \item  If M accepts w (i.e. it stops in an accepting state), M' should accept w'.
    \item  If M doesn't stop on w then let M' keep moving right (infinite loop). 
    \end{enumerate}
    Hence this reduction shows that $L_H$ is not recursive.
    
    
    \section*{Problem 2} A student unconvinced by the diagonalisation argument for proving $L_d$ is not e.e., approaches her Professor with the following doubt. Since the set $L_d$ is dependent on the ordering of the strings, what if, a different ordering $\mathcal{O}'$ is used? Why will the previous $L_d$ still continue to be a non r.e. set although it does not correspond to the diagonal in $\mathcal{O}'$? Can you answer her doubts? You can assume that both orderings can be computed using a TM.
    
    \textbf{Solution} : Given $L_d$ is NOT recursively enumerable set which corresponds to the diagonal in ordering $\mathcal{O}$. Let $L_d'$ be the NOT recursively enumerable set which corresponds to the diagonal in new ordering $\mathcal{O}'$. 
    
    Also let old set $L_d$, which does not correspond to the diagonal in the ordering $\mathcal{O}'$, be represented as $S$. Then we have to show that $S$ is still NOT recursively enumerable set although it does not correspond to the diagonal in $\mathcal{O}'$. We will prove the same by using reducibility.
    
    Given that both orderings can be computed using a TM. So the identity map $f:L_d\rightarrow S$ is computable which is defined by $f(w_i)=w_j'$ where $w_i$ is $i$th string in ordering $\mathcal{O}$ and $w_j'$ is $j$th string in ordering $\mathcal{O}'$ such that $w_i=w_j'$. Since this identity map is a bijection, so we have $L_d\leq_f S$. If we assume that $S$ is recursively enumerable, then $L_d\leq_f S$ implies that $L_d$ is also recursively enumerable which is a contradiction. 
    
    Hence $S$ must be NOT recursively enumerable which implies that $L_d$ remains NOT recursively enumerable set although it does not correspond to the diagonal in $\mathcal{O}'$.
    
\end{document}